\documentclass{article}
\usepackage{Stundenplan}
\usepackage[utf8]{inputenc}

\newcommand{\Ueberschrift}{1.Semester}
\renewcommand{\planname}{UHH -- Computing in Science - FB Biochemie}

\begin{document}
  \begin{plan}
    % Chemie Teil
    \VorlEins \montag {8}{15}{90}{Allgemeine Chemie\\\ \\MLK3, Großer Hörsaal}
    \Uebung \montag {10}{15}{90}{Allgemeine Chemie\\\ \\Sem.Raum S1  IAACh}
    \Seminar \montag {12}{15}{90}{CiS - Proseminar\\\ \\ZBH Raum 16}
    \VorlEins \donnerstag {12}{15}{90}{Allgemeine Chemie\\\ \\MLK3, Großer Hörsaal}
    \VorlEins \freitag {10}{15}{90}{Physikalische Chemie 1\\\ \\Hörsaal A Chemie}
    \Uebung \dienstag {12}{15}{45}{Physikalische Chemie 1\\\ \\Sem. Raum 160  IPhCh}
    % Informatik Teil
    \VorlZwei \mittwoch {14}{15}{90}{Softwareentwicklung 1\\\ \\Hörsaal A Chemie}
    \Uebung \dienstag {14}{00}{180}{Softwareentwicklung 1\\\ \\Ikum Raum D-017}
    % Mathe Teil
    \VorlDrei \mittwoch {8}{15}{90}{Mathematik für\\Physiker 1\\\ \\ESA A}
    \VorlDrei \freitag {8}{15}{90}{Mathematik für\\Physiker 1\\\ \\Erzwiss H}
    \Uebung \mittwoch {10}{15}{90}{Mathematik für\\Physiker 1\\\ \\Geomatikum Raum 435}
    \Tutorium \freitag {15}{15}{90}{Mathe 1 - Tutorium\\\ \\Ikum Raum F-009}
    
    % Test der einzelnen möglichen Farbgebungen in der Legende
    \Legende{\LVorlEins\LVorlZwei\LVorlDrei\LUebung\LTutorium\LSeminar\LPrivat}
  \end{plan}
\end{document}