\documentclass{article}
\usepackage{Stundenplan}
\usepackage[utf8]{inputenc}
\usepackage[hidelinks]{hyperref}

\newcommand{\Ueberschrift}{Mein X. Semester}
\renewcommand{\planname}{UHH -- Computing in Science - FB Biochemie}

\begin{document}
  % Allgemeine Synatax: \<Veranstaltung> \<tag> {<Uhrzeit(HH)>}{<Uhrzeit(MM)>}{<Dauer(Min)>}{<Veranstaltungname>}
  % Beispiel 1: Vorlesung 1 ist 08:15-09:45 und dauert 90 Min 
  %    -> "\VorlEins \montag {8}{15} {90}{Name - Vorlesung 1}"
  
  \begin{plan}
    \VorlEins \montag {8}{15} {90}{Name - Vorlesung 1}
    \VorlZwei \donnerstag {10}{15} {90}{Name - Vorlesung 2}
    \Deadline \dienstag {12}{00} {15}{Onlinetest - \href{www.test.de}{Link}}
    \VorlDrei \freitag {8}{15} {90}{Name - Vorlesung 2}
    \Uebung \mittwoch {10}{15} {90}{Übung zu Vorlesung 1\\\ \\Lösung:\\ \href{mailto:max.muster@informatik.uni-hamburg.de}{E-Mail senden}}
	\Tutorium \freitag {12}{30} {45}{\LaTeX - Tutorium\\\href{https://github.com/JamesJux/Latex-Stundenplan}{Material}}
    
    \Privat \dienstag{17}{00} {120}{Fitnessstudio}
    % Anzeigen der einzelnen möglichen Veranstaltungen in der Legende
    \Legende{\LVorlEins\LVorlZwei\LVorlDrei\LUebung\LTutorium\LSeminar\LPrivat}
  \end{plan}
\end{document}